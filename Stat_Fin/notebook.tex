
% Default to the notebook output style

    


% Inherit from the specified cell style.




    
\documentclass[11pt]{article}

    
    
    \usepackage[T1]{fontenc}
    % Nicer default font (+ math font) than Computer Modern for most use cases
    \usepackage{mathpazo}

    % Basic figure setup, for now with no caption control since it's done
    % automatically by Pandoc (which extracts ![](path) syntax from Markdown).
    \usepackage{graphicx}
    % We will generate all images so they have a width \maxwidth. This means
    % that they will get their normal width if they fit onto the page, but
    % are scaled down if they would overflow the margins.
    \makeatletter
    \def\maxwidth{\ifdim\Gin@nat@width>\linewidth\linewidth
    \else\Gin@nat@width\fi}
    \makeatother
    \let\Oldincludegraphics\includegraphics
    % Set max figure width to be 80% of text width, for now hardcoded.
    \renewcommand{\includegraphics}[1]{\Oldincludegraphics[width=.8\maxwidth]{#1}}
    % Ensure that by default, figures have no caption (until we provide a
    % proper Figure object with a Caption API and a way to capture that
    % in the conversion process - todo).
    \usepackage{caption}
    \DeclareCaptionLabelFormat{nolabel}{}
    \captionsetup{labelformat=nolabel}

    \usepackage{adjustbox} % Used to constrain images to a maximum size 
    \usepackage{xcolor} % Allow colors to be defined
    \usepackage{enumerate} % Needed for markdown enumerations to work
    \usepackage{geometry} % Used to adjust the document margins
    \usepackage{amsmath} % Equations
    \usepackage{amssymb} % Equations
    \usepackage{textcomp} % defines textquotesingle
    % Hack from http://tex.stackexchange.com/a/47451/13684:
    \AtBeginDocument{%
        \def\PYZsq{\textquotesingle}% Upright quotes in Pygmentized code
    }
    \usepackage{upquote} % Upright quotes for verbatim code
    \usepackage{eurosym} % defines \euro
    \usepackage[mathletters]{ucs} % Extended unicode (utf-8) support
    \usepackage[utf8x]{inputenc} % Allow utf-8 characters in the tex document
    \usepackage{fancyvrb} % verbatim replacement that allows latex
    \usepackage{grffile} % extends the file name processing of package graphics 
                         % to support a larger range 
    % The hyperref package gives us a pdf with properly built
    % internal navigation ('pdf bookmarks' for the table of contents,
    % internal cross-reference links, web links for URLs, etc.)
    \usepackage{hyperref}
    \usepackage{longtable} % longtable support required by pandoc >1.10
    \usepackage{booktabs}  % table support for pandoc > 1.12.2
    \usepackage[inline]{enumitem} % IRkernel/repr support (it uses the enumerate* environment)
    \usepackage[normalem]{ulem} % ulem is needed to support strikethroughs (\sout)
                                % normalem makes italics be italics, not underlines
    

    
    
    % Colors for the hyperref package
    \definecolor{urlcolor}{rgb}{0,.145,.698}
    \definecolor{linkcolor}{rgb}{.71,0.21,0.01}
    \definecolor{citecolor}{rgb}{.12,.54,.11}

    % ANSI colors
    \definecolor{ansi-black}{HTML}{3E424D}
    \definecolor{ansi-black-intense}{HTML}{282C36}
    \definecolor{ansi-red}{HTML}{E75C58}
    \definecolor{ansi-red-intense}{HTML}{B22B31}
    \definecolor{ansi-green}{HTML}{00A250}
    \definecolor{ansi-green-intense}{HTML}{007427}
    \definecolor{ansi-yellow}{HTML}{DDB62B}
    \definecolor{ansi-yellow-intense}{HTML}{B27D12}
    \definecolor{ansi-blue}{HTML}{208FFB}
    \definecolor{ansi-blue-intense}{HTML}{0065CA}
    \definecolor{ansi-magenta}{HTML}{D160C4}
    \definecolor{ansi-magenta-intense}{HTML}{A03196}
    \definecolor{ansi-cyan}{HTML}{60C6C8}
    \definecolor{ansi-cyan-intense}{HTML}{258F8F}
    \definecolor{ansi-white}{HTML}{C5C1B4}
    \definecolor{ansi-white-intense}{HTML}{A1A6B2}

    % commands and environments needed by pandoc snippets
    % extracted from the output of `pandoc -s`
    \providecommand{\tightlist}{%
      \setlength{\itemsep}{0pt}\setlength{\parskip}{0pt}}
    \DefineVerbatimEnvironment{Highlighting}{Verbatim}{commandchars=\\\{\}}
    % Add ',fontsize=\small' for more characters per line
    \newenvironment{Shaded}{}{}
    \newcommand{\KeywordTok}[1]{\textcolor[rgb]{0.00,0.44,0.13}{\textbf{{#1}}}}
    \newcommand{\DataTypeTok}[1]{\textcolor[rgb]{0.56,0.13,0.00}{{#1}}}
    \newcommand{\DecValTok}[1]{\textcolor[rgb]{0.25,0.63,0.44}{{#1}}}
    \newcommand{\BaseNTok}[1]{\textcolor[rgb]{0.25,0.63,0.44}{{#1}}}
    \newcommand{\FloatTok}[1]{\textcolor[rgb]{0.25,0.63,0.44}{{#1}}}
    \newcommand{\CharTok}[1]{\textcolor[rgb]{0.25,0.44,0.63}{{#1}}}
    \newcommand{\StringTok}[1]{\textcolor[rgb]{0.25,0.44,0.63}{{#1}}}
    \newcommand{\CommentTok}[1]{\textcolor[rgb]{0.38,0.63,0.69}{\textit{{#1}}}}
    \newcommand{\OtherTok}[1]{\textcolor[rgb]{0.00,0.44,0.13}{{#1}}}
    \newcommand{\AlertTok}[1]{\textcolor[rgb]{1.00,0.00,0.00}{\textbf{{#1}}}}
    \newcommand{\FunctionTok}[1]{\textcolor[rgb]{0.02,0.16,0.49}{{#1}}}
    \newcommand{\RegionMarkerTok}[1]{{#1}}
    \newcommand{\ErrorTok}[1]{\textcolor[rgb]{1.00,0.00,0.00}{\textbf{{#1}}}}
    \newcommand{\NormalTok}[1]{{#1}}
    
    % Additional commands for more recent versions of Pandoc
    \newcommand{\ConstantTok}[1]{\textcolor[rgb]{0.53,0.00,0.00}{{#1}}}
    \newcommand{\SpecialCharTok}[1]{\textcolor[rgb]{0.25,0.44,0.63}{{#1}}}
    \newcommand{\VerbatimStringTok}[1]{\textcolor[rgb]{0.25,0.44,0.63}{{#1}}}
    \newcommand{\SpecialStringTok}[1]{\textcolor[rgb]{0.73,0.40,0.53}{{#1}}}
    \newcommand{\ImportTok}[1]{{#1}}
    \newcommand{\DocumentationTok}[1]{\textcolor[rgb]{0.73,0.13,0.13}{\textit{{#1}}}}
    \newcommand{\AnnotationTok}[1]{\textcolor[rgb]{0.38,0.63,0.69}{\textbf{\textit{{#1}}}}}
    \newcommand{\CommentVarTok}[1]{\textcolor[rgb]{0.38,0.63,0.69}{\textbf{\textit{{#1}}}}}
    \newcommand{\VariableTok}[1]{\textcolor[rgb]{0.10,0.09,0.49}{{#1}}}
    \newcommand{\ControlFlowTok}[1]{\textcolor[rgb]{0.00,0.44,0.13}{\textbf{{#1}}}}
    \newcommand{\OperatorTok}[1]{\textcolor[rgb]{0.40,0.40,0.40}{{#1}}}
    \newcommand{\BuiltInTok}[1]{{#1}}
    \newcommand{\ExtensionTok}[1]{{#1}}
    \newcommand{\PreprocessorTok}[1]{\textcolor[rgb]{0.74,0.48,0.00}{{#1}}}
    \newcommand{\AttributeTok}[1]{\textcolor[rgb]{0.49,0.56,0.16}{{#1}}}
    \newcommand{\InformationTok}[1]{\textcolor[rgb]{0.38,0.63,0.69}{\textbf{\textit{{#1}}}}}
    \newcommand{\WarningTok}[1]{\textcolor[rgb]{0.38,0.63,0.69}{\textbf{\textit{{#1}}}}}
    
    
    % Define a nice break command that doesn't care if a line doesn't already
    % exist.
    \def\br{\hspace*{\fill} \\* }
    % Math Jax compatability definitions
    \def\gt{>}
    \def\lt{<}
    % Document parameters
    \title{hw5\_fy2232}
    
    
    

    % Pygments definitions
    
\makeatletter
\def\PY@reset{\let\PY@it=\relax \let\PY@bf=\relax%
    \let\PY@ul=\relax \let\PY@tc=\relax%
    \let\PY@bc=\relax \let\PY@ff=\relax}
\def\PY@tok#1{\csname PY@tok@#1\endcsname}
\def\PY@toks#1+{\ifx\relax#1\empty\else%
    \PY@tok{#1}\expandafter\PY@toks\fi}
\def\PY@do#1{\PY@bc{\PY@tc{\PY@ul{%
    \PY@it{\PY@bf{\PY@ff{#1}}}}}}}
\def\PY#1#2{\PY@reset\PY@toks#1+\relax+\PY@do{#2}}

\expandafter\def\csname PY@tok@gd\endcsname{\def\PY@tc##1{\textcolor[rgb]{0.63,0.00,0.00}{##1}}}
\expandafter\def\csname PY@tok@gu\endcsname{\let\PY@bf=\textbf\def\PY@tc##1{\textcolor[rgb]{0.50,0.00,0.50}{##1}}}
\expandafter\def\csname PY@tok@gt\endcsname{\def\PY@tc##1{\textcolor[rgb]{0.00,0.27,0.87}{##1}}}
\expandafter\def\csname PY@tok@gs\endcsname{\let\PY@bf=\textbf}
\expandafter\def\csname PY@tok@gr\endcsname{\def\PY@tc##1{\textcolor[rgb]{1.00,0.00,0.00}{##1}}}
\expandafter\def\csname PY@tok@cm\endcsname{\let\PY@it=\textit\def\PY@tc##1{\textcolor[rgb]{0.25,0.50,0.50}{##1}}}
\expandafter\def\csname PY@tok@vg\endcsname{\def\PY@tc##1{\textcolor[rgb]{0.10,0.09,0.49}{##1}}}
\expandafter\def\csname PY@tok@vi\endcsname{\def\PY@tc##1{\textcolor[rgb]{0.10,0.09,0.49}{##1}}}
\expandafter\def\csname PY@tok@vm\endcsname{\def\PY@tc##1{\textcolor[rgb]{0.10,0.09,0.49}{##1}}}
\expandafter\def\csname PY@tok@mh\endcsname{\def\PY@tc##1{\textcolor[rgb]{0.40,0.40,0.40}{##1}}}
\expandafter\def\csname PY@tok@cs\endcsname{\let\PY@it=\textit\def\PY@tc##1{\textcolor[rgb]{0.25,0.50,0.50}{##1}}}
\expandafter\def\csname PY@tok@ge\endcsname{\let\PY@it=\textit}
\expandafter\def\csname PY@tok@vc\endcsname{\def\PY@tc##1{\textcolor[rgb]{0.10,0.09,0.49}{##1}}}
\expandafter\def\csname PY@tok@il\endcsname{\def\PY@tc##1{\textcolor[rgb]{0.40,0.40,0.40}{##1}}}
\expandafter\def\csname PY@tok@go\endcsname{\def\PY@tc##1{\textcolor[rgb]{0.53,0.53,0.53}{##1}}}
\expandafter\def\csname PY@tok@cp\endcsname{\def\PY@tc##1{\textcolor[rgb]{0.74,0.48,0.00}{##1}}}
\expandafter\def\csname PY@tok@gi\endcsname{\def\PY@tc##1{\textcolor[rgb]{0.00,0.63,0.00}{##1}}}
\expandafter\def\csname PY@tok@gh\endcsname{\let\PY@bf=\textbf\def\PY@tc##1{\textcolor[rgb]{0.00,0.00,0.50}{##1}}}
\expandafter\def\csname PY@tok@ni\endcsname{\let\PY@bf=\textbf\def\PY@tc##1{\textcolor[rgb]{0.60,0.60,0.60}{##1}}}
\expandafter\def\csname PY@tok@nl\endcsname{\def\PY@tc##1{\textcolor[rgb]{0.63,0.63,0.00}{##1}}}
\expandafter\def\csname PY@tok@nn\endcsname{\let\PY@bf=\textbf\def\PY@tc##1{\textcolor[rgb]{0.00,0.00,1.00}{##1}}}
\expandafter\def\csname PY@tok@no\endcsname{\def\PY@tc##1{\textcolor[rgb]{0.53,0.00,0.00}{##1}}}
\expandafter\def\csname PY@tok@na\endcsname{\def\PY@tc##1{\textcolor[rgb]{0.49,0.56,0.16}{##1}}}
\expandafter\def\csname PY@tok@nb\endcsname{\def\PY@tc##1{\textcolor[rgb]{0.00,0.50,0.00}{##1}}}
\expandafter\def\csname PY@tok@nc\endcsname{\let\PY@bf=\textbf\def\PY@tc##1{\textcolor[rgb]{0.00,0.00,1.00}{##1}}}
\expandafter\def\csname PY@tok@nd\endcsname{\def\PY@tc##1{\textcolor[rgb]{0.67,0.13,1.00}{##1}}}
\expandafter\def\csname PY@tok@ne\endcsname{\let\PY@bf=\textbf\def\PY@tc##1{\textcolor[rgb]{0.82,0.25,0.23}{##1}}}
\expandafter\def\csname PY@tok@nf\endcsname{\def\PY@tc##1{\textcolor[rgb]{0.00,0.00,1.00}{##1}}}
\expandafter\def\csname PY@tok@si\endcsname{\let\PY@bf=\textbf\def\PY@tc##1{\textcolor[rgb]{0.73,0.40,0.53}{##1}}}
\expandafter\def\csname PY@tok@s2\endcsname{\def\PY@tc##1{\textcolor[rgb]{0.73,0.13,0.13}{##1}}}
\expandafter\def\csname PY@tok@nt\endcsname{\let\PY@bf=\textbf\def\PY@tc##1{\textcolor[rgb]{0.00,0.50,0.00}{##1}}}
\expandafter\def\csname PY@tok@nv\endcsname{\def\PY@tc##1{\textcolor[rgb]{0.10,0.09,0.49}{##1}}}
\expandafter\def\csname PY@tok@s1\endcsname{\def\PY@tc##1{\textcolor[rgb]{0.73,0.13,0.13}{##1}}}
\expandafter\def\csname PY@tok@dl\endcsname{\def\PY@tc##1{\textcolor[rgb]{0.73,0.13,0.13}{##1}}}
\expandafter\def\csname PY@tok@ch\endcsname{\let\PY@it=\textit\def\PY@tc##1{\textcolor[rgb]{0.25,0.50,0.50}{##1}}}
\expandafter\def\csname PY@tok@m\endcsname{\def\PY@tc##1{\textcolor[rgb]{0.40,0.40,0.40}{##1}}}
\expandafter\def\csname PY@tok@gp\endcsname{\let\PY@bf=\textbf\def\PY@tc##1{\textcolor[rgb]{0.00,0.00,0.50}{##1}}}
\expandafter\def\csname PY@tok@sh\endcsname{\def\PY@tc##1{\textcolor[rgb]{0.73,0.13,0.13}{##1}}}
\expandafter\def\csname PY@tok@ow\endcsname{\let\PY@bf=\textbf\def\PY@tc##1{\textcolor[rgb]{0.67,0.13,1.00}{##1}}}
\expandafter\def\csname PY@tok@sx\endcsname{\def\PY@tc##1{\textcolor[rgb]{0.00,0.50,0.00}{##1}}}
\expandafter\def\csname PY@tok@bp\endcsname{\def\PY@tc##1{\textcolor[rgb]{0.00,0.50,0.00}{##1}}}
\expandafter\def\csname PY@tok@c1\endcsname{\let\PY@it=\textit\def\PY@tc##1{\textcolor[rgb]{0.25,0.50,0.50}{##1}}}
\expandafter\def\csname PY@tok@fm\endcsname{\def\PY@tc##1{\textcolor[rgb]{0.00,0.00,1.00}{##1}}}
\expandafter\def\csname PY@tok@o\endcsname{\def\PY@tc##1{\textcolor[rgb]{0.40,0.40,0.40}{##1}}}
\expandafter\def\csname PY@tok@kc\endcsname{\let\PY@bf=\textbf\def\PY@tc##1{\textcolor[rgb]{0.00,0.50,0.00}{##1}}}
\expandafter\def\csname PY@tok@c\endcsname{\let\PY@it=\textit\def\PY@tc##1{\textcolor[rgb]{0.25,0.50,0.50}{##1}}}
\expandafter\def\csname PY@tok@mf\endcsname{\def\PY@tc##1{\textcolor[rgb]{0.40,0.40,0.40}{##1}}}
\expandafter\def\csname PY@tok@err\endcsname{\def\PY@bc##1{\setlength{\fboxsep}{0pt}\fcolorbox[rgb]{1.00,0.00,0.00}{1,1,1}{\strut ##1}}}
\expandafter\def\csname PY@tok@mb\endcsname{\def\PY@tc##1{\textcolor[rgb]{0.40,0.40,0.40}{##1}}}
\expandafter\def\csname PY@tok@ss\endcsname{\def\PY@tc##1{\textcolor[rgb]{0.10,0.09,0.49}{##1}}}
\expandafter\def\csname PY@tok@sr\endcsname{\def\PY@tc##1{\textcolor[rgb]{0.73,0.40,0.53}{##1}}}
\expandafter\def\csname PY@tok@mo\endcsname{\def\PY@tc##1{\textcolor[rgb]{0.40,0.40,0.40}{##1}}}
\expandafter\def\csname PY@tok@kd\endcsname{\let\PY@bf=\textbf\def\PY@tc##1{\textcolor[rgb]{0.00,0.50,0.00}{##1}}}
\expandafter\def\csname PY@tok@mi\endcsname{\def\PY@tc##1{\textcolor[rgb]{0.40,0.40,0.40}{##1}}}
\expandafter\def\csname PY@tok@kn\endcsname{\let\PY@bf=\textbf\def\PY@tc##1{\textcolor[rgb]{0.00,0.50,0.00}{##1}}}
\expandafter\def\csname PY@tok@cpf\endcsname{\let\PY@it=\textit\def\PY@tc##1{\textcolor[rgb]{0.25,0.50,0.50}{##1}}}
\expandafter\def\csname PY@tok@kr\endcsname{\let\PY@bf=\textbf\def\PY@tc##1{\textcolor[rgb]{0.00,0.50,0.00}{##1}}}
\expandafter\def\csname PY@tok@s\endcsname{\def\PY@tc##1{\textcolor[rgb]{0.73,0.13,0.13}{##1}}}
\expandafter\def\csname PY@tok@kp\endcsname{\def\PY@tc##1{\textcolor[rgb]{0.00,0.50,0.00}{##1}}}
\expandafter\def\csname PY@tok@w\endcsname{\def\PY@tc##1{\textcolor[rgb]{0.73,0.73,0.73}{##1}}}
\expandafter\def\csname PY@tok@kt\endcsname{\def\PY@tc##1{\textcolor[rgb]{0.69,0.00,0.25}{##1}}}
\expandafter\def\csname PY@tok@sc\endcsname{\def\PY@tc##1{\textcolor[rgb]{0.73,0.13,0.13}{##1}}}
\expandafter\def\csname PY@tok@sb\endcsname{\def\PY@tc##1{\textcolor[rgb]{0.73,0.13,0.13}{##1}}}
\expandafter\def\csname PY@tok@sa\endcsname{\def\PY@tc##1{\textcolor[rgb]{0.73,0.13,0.13}{##1}}}
\expandafter\def\csname PY@tok@k\endcsname{\let\PY@bf=\textbf\def\PY@tc##1{\textcolor[rgb]{0.00,0.50,0.00}{##1}}}
\expandafter\def\csname PY@tok@se\endcsname{\let\PY@bf=\textbf\def\PY@tc##1{\textcolor[rgb]{0.73,0.40,0.13}{##1}}}
\expandafter\def\csname PY@tok@sd\endcsname{\let\PY@it=\textit\def\PY@tc##1{\textcolor[rgb]{0.73,0.13,0.13}{##1}}}

\def\PYZbs{\char`\\}
\def\PYZus{\char`\_}
\def\PYZob{\char`\{}
\def\PYZcb{\char`\}}
\def\PYZca{\char`\^}
\def\PYZam{\char`\&}
\def\PYZlt{\char`\<}
\def\PYZgt{\char`\>}
\def\PYZsh{\char`\#}
\def\PYZpc{\char`\%}
\def\PYZdl{\char`\$}
\def\PYZhy{\char`\-}
\def\PYZsq{\char`\'}
\def\PYZdq{\char`\"}
\def\PYZti{\char`\~}
% for compatibility with earlier versions
\def\PYZat{@}
\def\PYZlb{[}
\def\PYZrb{]}
\makeatother


    % Exact colors from NB
    \definecolor{incolor}{rgb}{0.0, 0.0, 0.5}
    \definecolor{outcolor}{rgb}{0.545, 0.0, 0.0}



    
    % Prevent overflowing lines due to hard-to-break entities
    \sloppy 
    % Setup hyperref package
    \hypersetup{
      breaklinks=true,  % so long urls are correctly broken across lines
      colorlinks=true,
      urlcolor=urlcolor,
      linkcolor=linkcolor,
      citecolor=citecolor,
      }
    % Slightly bigger margins than the latex defaults
    
    \geometry{verbose,tmargin=1in,bmargin=1in,lmargin=1in,rmargin=1in}
    
    

    \begin{document}
    
    
    \maketitle
    
    

    
     \#

STATISTICAL METHODS IN FINANCE

\#

HW5

Fan Yang

UNI: fy2232

02/28/2018

    \hypertarget{problem-1}{%
\section{Problem 1}\label{problem-1}}

    \textbf{\emph{Show that the following inequalities are always true for
any copula function \(C\): \[C^−(u, v) \leq C(u, v) \leq C^+(u, v).\]}}
\textbf{\emph{Recall that \(C^−(u, v) = \max(u + v − 1, 0)\) and
\(C^+(u, v) = \min(u, v)\).}}

    \hypertarget{i-cu-v-leq-cu-v}{%
\subsubsection{\texorpdfstring{(i)
\(C^−(u, v) \leq C(u, v)\)}{(i) C\^{}−(u, v) \textbackslash{}leq C(u, v)}}\label{i-cu-v-leq-cu-v}}

    By 2-increasing, for all \(u_1\leq u_2,v_1\leq v_2\),
\[C(u_2,v_2)-C(u_1,v_2)-C(u_2,v_1)+C(u_1,v_1)\geq 0\] Let \(u_2=v_2=1\)
and \(u_1= u, ~ v_1=v\), then \[C(1,1)-C(u,1)-C(1,v)+C(u,v)\geq 0\\
1-u-v+C(u,v)\geq 0\\
C(u,v)\geq u+v-1\] While \$C(u,v)\geq 0 \$, therefore
\(\max(u + v − 1, 0) \leq C(u, v)\) that's to say,
\(C^−(u, v) \leq C(u, v)\).

    \hypertarget{ii-cu-v-leq-cu-v}{%
\subsubsection{\texorpdfstring{(ii)
\(C(u, v) \leq C^+(u, v)\)}{(ii) C(u, v) \textbackslash{}leq C\^{}+(u, v)}}\label{ii-cu-v-leq-cu-v}}

    By 2-increasing, for all \(u_1\leq u_2,v_1\leq v_2\),
\[C(u_2,v_2)-C(u_1,v_2)-C(u_2,v_1)+C(u_1,v_1)\geq 0\] Let \(u_2=1\) and
\(u_1=u\) and \(v_1= 0, ~ v_2=v\), then
\[C(1,v)-C(u,v)-C(1,0)+C(u,0)\geq 0\\
C(1,v)-C(u,v)\geq 0\\
v-C(u,v)\geq 0\\
C(u,v)\leq v\] Let \(u_2=u\) and \(u_1=0\) and \(v_1= v, ~ v_2=1\), then
\[C(u,1)-C(0,1)-C(u,v)+C(0,v)\geq 0\\
C(u,1)-C(u,v)\geq 0\\
u-C(u,v)\geq 0\\
C(u,v)\leq u\] From all above, we have \(C(u, v) \leq\min(u, v)\) that's
to say, \(C(u, v)\leq C^+(u, v)\).

    \hypertarget{problem-2}{%
\section{Problem 2}\label{problem-2}}

    \textbf{\emph{Kendall's tau rank correlation between \(X\) and \(Y\) is
\(0.55\). Both \(X\) and \(Y\) are strictly positive random variables.
What is Kendall's tau between \(X\) and \(1/Y\) ? What is the Kendall's
tau between \(1/X\) and \(1/Y\) ?}}

    \textbackslash{}begin\{split\}
\rho\emph{\tau(X,Y)\&=E{[}\text{sign}\{(X-X\^{}\emph{)(Y-Y\^{}})\}{]}\textbackslash{}
\&=\int\int\textsubscript{\text{sign}\{(x-x\^{}\emph{)(y-y\^{}})\}}p}\{X,Y\}(x,y)\textsubscript{dx}dy
\textbackslash{}end\{split\}

    \hypertarget{x-and-1y}{%
\subsubsection{\texorpdfstring{\(X\) and
\(1/Y\)}{X and 1/Y}}\label{x-and-1y}}

    Notice that X and Y are strictly positive.
\textbackslash{}begin\{split\}
\rho\_\tau(X,1/Y)\&=E{[}\text{sign}\{(X-X\^{}\emph{)(1/Y-1/Y\^{}})\}{]}\textbackslash{}
\&=\int\int\textasciitilde{}\text{sign}\{(x-x\^{}*)(\frac{1}{y}-\frac{1}{y^*})\}\textsubscript{p\_\{X,Y\}(x,y)}dx\textasciitilde{}dy
\textbackslash{}end\{split\} Note that for any positive \(y\) and
\(y^*\), \(\text{sign}\{(x-x^*)(\frac{1}{y}-\frac{1}{y^*})\}\) and
\(\text{sign}\{(x-x^*)(y-y^*)\}\) always have opposite signs. which
means
\(\text{sign}\{(x-x^*)(\frac{1}{y}-\frac{1}{y^*})\} = -~\text{sign}\{(x-x^*)(y-y^*)\}\).
So, \textbackslash{}begin\{split\}
\rho\_\tau(X,1/Y)\&=E{[}\text{sign}\{(X-X\^{}\emph{)(1/Y-1/Y\^{}})\}{]}\textbackslash{}
\&=\int\int\textasciitilde{}\text{sign}\{(x-x\^{}*)(\frac{1}{y}-\frac{1}{y^*})\}\textsubscript{p\_\{X,Y\}(x,y)}dx\textsubscript{dy\textbackslash{}
\&=\int\int}-\textsubscript{\text{sign}\{(x-x\^{}\emph{)(y-y\^{}})\}}p\_\{X,Y\}(x,y)\textsubscript{dx}dy\textbackslash{}
\&=-\rho\_\tau(X,Y)\textbackslash{} \&=-0.55
\textbackslash{}end\{split\}

    \hypertarget{x-and-1y}{%
\subsubsection{\texorpdfstring{\(1/X\) and
\(1/Y\)}{1/X and 1/Y}}\label{x-and-1y}}

    \textbackslash{}begin\{split\}
\rho\emph{\tau(1/X,1/Y)\&=E{[}\text{sign}\{(1/X-1/X\^{}\emph{)(1/Y-1/Y\^{}})\}{]}\textbackslash{}
\&=\int\int\textsubscript{\text{sign}\{(\frac{1}{x}-\frac{1}{x^*})(\frac{1}{y}-\frac{1}{y^*})\}}p}\{X,Y\}(x,y)\textsubscript{dx}dy
\textbackslash{}end\{split\} Note that for any positive \$ x\$, \(x^*\),
\(y\) and \(y^*\), \(\text{sign}\{\frac{1}{y}-\frac{1}{y^*}\}\) and
\(\text{sign}\{y-y^*\}\) always have opposite signs. And
\(\text{sign}\{\frac{1}{x}-\frac{1}{x^*}\}\) and
\(\text{sign}\{x-x^*\}\) always have opposite signs. Which indicates
that
\(\text{sign}\{(\frac{1}{x}-\frac{1}{x^*})(\frac{1}{y}-\frac{1}{y^*})\} = ~\text{sign}\{(x-x^*)(y-y^*)\}\).
So, \textbackslash{}begin\{split\}
\rho\emph{\tau(X,1/Y)\&=E{[}\text{sign}\{(X-X\^{}\emph{)(1/Y-1/Y\^{}})\}{]}\textbackslash{}
\&=\int\int\textsubscript{\text{sign}\{(\frac{1}{x}-\frac{1}{x^*})(\frac{1}{y}-\frac{1}{y^*})\}}p}\{X,Y\}(x,y)\textsubscript{dx}dy\textbackslash{}
\&=\int\int\textsubscript{\text{sign}\{(x-x\^{}\emph{)(y-y\^{}})\}}p\_\{X,Y\}(x,y)\textsubscript{dx}dy\textbackslash{}
\&=\rho\_\tau(X,Y)\textbackslash{} \&=0.55 \textbackslash{}end\{split\}

    \hypertarget{problem-3}{%
\section{Problem 3}\label{problem-3}}

    \textbf{\emph{Suppose that \(X\) is Uniform(0,1) and \(Y = X^2\). Then
the Spearman rank correlation and the Kendall's tau between \(X\) and
\(Y\) will both equal 1, but the Pearson correlation between \(X\) and
\(Y\) will be less than 1. Explain why this is the case.}}

    \textbackslash{}begin\{split\}
F\_Y(y)\&=P(Y\leq y)=P(X\^{}2\leq y)\textbackslash{}
\&=P(X\leq \sqrt{y})\textbackslash{} \&=\sqrt{y}\textbackslash{} f\_Y(y)
\&= \frac{d F_Y(y)}{dy} = \frac{1}{2\sqrt{y}}. \textasciitilde{}
\textasciitilde{}y\in [0,1]
\textbackslash{}end\{split\}

    \hypertarget{spearman-rank-correlation}{%
\subsubsection{Spearman rank
correlation}\label{spearman-rank-correlation}}

    \textbackslash{}begin\{split\}
\rho\_S(X,Y)\&=\text{Corr}\{F\_X(X),F\_Y(Y)\}\textbackslash{}
\&=\frac{\text{Cov}\{F_X(X),F_Y(Y)\}}{\sqrt{\text{Var}(F_X(X))\text{Var}(F_Y(Y))}}\textbackslash{}
\&=\frac{\text{Cov}\{X,\sqrt{Y}\}}{\sqrt{\text{Var}(X)\text{Var}(\sqrt{Y})}}\textbackslash{}
\&=\frac{\text{Cov}\{X,\sqrt{X^2}\}}{\sqrt{\text{Var}(X)\text{Var}(\sqrt{X^2})}}\textbackslash{}
\&=\frac{\text{Cov}\{X,X\}}{\sqrt{\text{Var}(X)\text{Var}(X)}}\textbackslash{}
\&=\frac{\text{Var}(X)}{\text{Var}(X)}\textbackslash{}
\&=1\textbackslash{} \textbackslash{}end\{split\}

    \hypertarget{kendalls-tau}{%
\subsubsection{Kendall's tau}\label{kendalls-tau}}

    Since \(Y=X^2\), for any \(x_i>x_j>0\), we have \(x_i^2>x_j^2\), which
is \(y_i>y_j\). In short, \(x>x^*\) indicates \(y>y^*\). Therefore,
\(\text{sign}\{(x-x^*)(y-y^*)\}\) is always 1.
\textbackslash{}begin\{split\}
\rho\emph{\tau(X,Y)\&=E{[}\text{sign}\{(X-X\^{}\emph{)(Y-Y\^{}})\}{]}\textbackslash{}
\&=\int\int\textsubscript{\text{sign}\{(x-x\^{}\emph{)(y-y\^{}})\}}p}\{X,Y\}(x,y)\textsubscript{dx}dy\textbackslash{}
\&=\int\int\textsubscript{p\_\{X,Y\}(x,y)}dx\textasciitilde{}dy\textbackslash{}
\&=1\textbackslash{} \textbackslash{}end\{split\}

    \hypertarget{pearson-correlation}{%
\subsubsection{Pearson correlation}\label{pearson-correlation}}

    \(X\sim Uniform(0,1)\), so \(E(X)=0.5\)
\(E(Y) = \int_0^1yf_Y(y)dy=\int_0^1y\frac{1}{2\sqrt{y}}dy=\frac{1}{3}\)

\begin{split}
\rho(X,Y)&=\text{Corr}(X,Y)\\
&=\frac{\text{Cov}(X,Y)}{\sqrt{\text{Var}(X)\text{Var}(Y)}}\\
\text{Cov}(X,Y)&=E[(X-E(X))(Y-E(Y))]\\
&=E[(X-\frac{1}{2})(X^2-\frac{1}{3})]\\
&=E[X^3-\frac{1}{2}X^2-\frac{1}{3}X+\frac{1}{6}]\\
&=E(X^3)-\frac{1}{2}E(X^2)-\frac{1}{3}E(X)+\frac{1}{6}\\
\end{split}

\(E(X^3) = \int_0^1x^3dx=\frac{1}{4}\)
\(E(X^2) = \int_0^1x^2dx=\frac{1}{3}\) \(Var(X) = \frac{1}{12}\)
\(E(Y^2) = \int_0^1y^2\frac{1}{2\sqrt{y}}dy=\frac{1}{5}\)
\(Var(Y) = E(Y^2)-E^2(Y)=\frac{1}{5}-(\frac{1}{3})^2=\frac{4}{45}\) So,

\begin{split}
\rho(X,Y)&=
\frac{E(X^3)-\frac{1}{2}E(X^2)-\frac{1}{3}E(X)+\frac{1}{6}}{\sqrt{\text{Var}(X)\text{Var}(Y)}}\\
&=\frac{\frac{1}{4}-\frac{1}{2}\times\frac{1}{3}-\frac{1}{3}\times\frac{1}{2}+\frac{1}{6}}{\sqrt{\frac{1}{12}\times\frac{4}{45}}}\\
&=\frac{\frac{1}{12}}{\sqrt{\frac{1}{135}}}\\
&=0.9682<1\\
\end{split}

    \hypertarget{problem-4}{%
\section{Problem 4}\label{problem-4}}

    *** (Optional) Recall that the bivariate Gumbel copula takes the form
\[C_{\alpha}(u,v) = \exp\{−[(−\log u)^\alpha +(−\log v)^\alpha]^{\frac{1}{\alpha}} \},
\] where \(\alpha \in [1,\infty)\). Show that, as
\(\alpha \rightarrow \infty\),
\(C_\alpha(u,v) \rightarrow C^+(u,v) = \min(u,v)\).***

    Denote \(a = −\log u \in (0,\infty)\) and \(b = −\log v \in (0,\infty)\)
\textbf{Without loss of generality}, let's suppose \(u\leq v\), then
\(a\geq b >0\) Let's consider
\(−[(−\log u)^\alpha +(−\log v)^\alpha]^{\frac{1}{\alpha}}\) which is
\(−(a^\alpha +b^\alpha)^{\frac{1}{\alpha}}\) first. Since \(a\geq b\),
we have \textbackslash{}begin\{split\}
−(a\textsuperscript{\alpha +a}\alpha)\^{}\{\frac{1}{\alpha}\}
\&\leq −(a\textsuperscript{\alpha +b}\alpha)\^{}\{\frac{1}{\alpha}\}
\&\textless{}
−(a\textsuperscript{\alpha )}\{\frac{1}{\alpha}\}\textbackslash{}
−(2a\textsuperscript{\alpha)}\{\frac{1}{\alpha}\}
\&\leq −(a\textsuperscript{\alpha +b}\alpha)\^{}\{\frac{1}{\alpha}\}
\&\textless{}
−(a\textsuperscript{\alpha )}\{\frac{1}{\alpha}\}\textbackslash{}
\lim\emph{\{\alpha\rightarrow\infty\}−(2a\textsuperscript{\alpha)}\{\frac{1}{\alpha}\}
\&\leq \lim}\{\alpha\rightarrow\infty\}−(a\textsuperscript{\alpha +b}\alpha)\^{}\{\frac{1}{\alpha}\}
\&\textless{}
\lim\emph{\{\alpha\rightarrow\infty\}−(a\textsuperscript{\alpha )}\{\frac{1}{\alpha}\}\textbackslash{}
-a
\&\leq \lim}\{\alpha\rightarrow\infty\}−(a\textsuperscript{\alpha +b}\alpha)\^{}\{\frac{1}{\alpha}\}
\&\textless{} -a\textbackslash{} \textbackslash{}end\{split\} Therefore,
\(\lim\limits_{\alpha\rightarrow\infty}−(a^\alpha +b^\alpha)^{\frac{1}{\alpha}} = -a\)
which is
\(\lim\limits_{\alpha\rightarrow\infty}−[(−\log u)^\alpha +(−\log v)^\alpha]^{\frac{1}{\alpha}} = \log u\)
So,
\(\lim\limits_{\alpha\rightarrow\infty}\exp\{−[(−\log u)^\alpha +(−\log v)^\alpha]^{\frac{1}{\alpha}} \} = \exp (\log u)=u\)
Under our assumption \(u\leq v\), so,
\(\lim\limits_{\alpha\rightarrow\infty}C_\alpha(u,v) = \min(u,v) = C^+(u,v)\)

    \hypertarget{problem-5}{%
\section{Problem 5}\label{problem-5}}

    \textbf{\emph{We have two \(B\)-rated bonds, with one-year default
probability at 3.46\%. Suppose that the interest rate is 4\%, and their
default times satisfy the Gaussian copula with \(\rho = 0.5\), calculate
the following expected present values, i.e.~fair prices.}}
\textgreater{}\textbf{\emph{(a)}} What is the fair price of a
first-to-default swap which pays \$1,000,000 if at least one of them
defaults by the end of the first year? \textbf{\emph{(b)}} How about a
second-to-default swap which pays \$1,000,000 if both default in the
first year?

    \hypertarget{a}{%
\subsection{(a)}\label{a}}

\textbackslash{}begin\{split\} FTD \&=
e\^{}\{-rT\}{[}F\_1(T)+F\_2(T)-C(F\_1(T),F\_2(T)){]}\textbackslash{} FTD
\&=1000000\times e\^{}\{-0.04\times1\}\times[0.0346+0.0346-C(0.0346,0.0346)]\textbackslash{}
\textbackslash{}end\{split\}

    \begin{Verbatim}[commandchars=\\\{\}]
{\color{incolor}In [{\color{incolor}1}]:} \PY{k+kn}{library}\PY{p}{(}mvtnorm\PY{p}{)}
        pmvnorm\PY{p}{(}lower \PY{o}{=} \PY{k+kt}{c}\PY{p}{(}\PY{o}{\PYZhy{}}\PY{k+kc}{Inf}\PY{p}{,}\PY{o}{\PYZhy{}}\PY{k+kc}{Inf}\PY{p}{)}\PY{p}{,} upper \PY{o}{=} \PY{k+kt}{c}\PY{p}{(}qnorm\PY{p}{(}\PY{l+m}{0.0346}\PY{p}{)}\PY{p}{,}qnorm\PY{p}{(}\PY{l+m}{0.0346}\PY{p}{)}\PY{p}{)}\PY{p}{,}
                sigma \PY{o}{=} \PY{k+kt}{matrix}\PY{p}{(} \PY{k+kt}{c}\PY{p}{(}\PY{l+m}{1}\PY{p}{,}\PY{l+m}{0.5}\PY{p}{,}\PY{l+m}{0.5}\PY{p}{,}\PY{l+m}{1}\PY{p}{)}\PY{p}{,}\PY{l+m}{2}\PY{p}{,}\PY{l+m}{2}\PY{p}{)}\PY{p}{)}
\end{Verbatim}


    0.00727678978541412

    
    Gaussian copula with \(\rho = 0.5\)

\begin{split}
FTD &=1000000\times e^{-0.04\times1}\times[0.0346+0.0346-C(0.0346,0.0346)]\\
&=1000000\times e^{-0.04\times1}\times[0.0346+0.0346-0.00727679]\\
&=59495.166\\
\end{split}

    \hypertarget{b}{%
\subsection{(b)}\label{b}}

\textbackslash{}begin\{split\} LTD \&=
e\^{}\{-rT\}C(F\_1(T),F\_2(T))\textbackslash{} LTD
\&=1000000\times e\^{}\{-0.04\times1\}\times C(0.0346,0.0346)\textbackslash{}
\textbackslash{}end\{split\}

    Gaussian copula with \(\rho = 0.5\)

\begin{split}
LTD &=1000000\times e^{-0.04\times1}\times[0.0346+0.0346-C(0.0346,0.0346)]\\
&=1000000\times e^{-0.04\times1}\times0.00727679\\
&=6991.46298\\
\end{split}

    \hypertarget{problem-6}{%
\section{Problem 6}\label{problem-6}}

    \textbf{\emph{Suppose that we have two bonds A and B. Denote by \(T_A\)
and \(T_B\) their respective default times (in year). Suppose that
\(T_A\) follows exponential distribution with hazard
\(\lambda_A = 0.01\) (i.e. \(P (T_A \geq t) = e^{−\lambda_A t}\)) and
\(T_B\) follows exponential with hazard \(\lambda_B = 0.02\). Suppose
that jointly they satisfy the Gumbel copula with \(\alpha = 2\). Find
the probabilities that (i) both will default by the end of the first
year; (ii) at least one will default by the end of the first year.}}

    \hypertarget{i}{%
\subsection{(i)}\label{i}}

\[g^{-1}_\alpha(u)=\exp(-u^{1/\alpha})=\exp(-\sqrt{u})\\
\tag{6.1}\] \[
P(\tau_A\leq T,\tau_B\leq T)
=C(F_A(T),F_B(T))\\
\tag{6.2}\] \textbackslash{}begin\{split\}
F\_A(T)\&=P(T\_A\leq T)\textbackslash{} \&=
1-P(T\_A\geq T)\textbackslash{}
\&=1-e\textsuperscript{\{-\lambda\emph{AT\}\textbackslash{}
\&=1-e\^{}\{-0.01\}\textbackslash{}
F\_B(T)\&=P(T\_B\leq T)\textbackslash{} \&=
1-P(T\_B\geq T)\textbackslash{}
\&=1-e\textsuperscript{\{-\lambda\emph{BT\}\textbackslash{}
\&=1-e\textsuperscript{\{-0.02\}\textbackslash{}\textbackslash{}
P(\tau\emph{A\leq 1,\tau\emph{B\leq 1)\&=C\_g(F\_A(1),F\_B(1))\textbackslash{}
\&=g\textsuperscript{\{-1\}\emph{\alpha(g}\alpha(F\_A(1))+g\_\alpha(F\_A(1)))\textbackslash{}
\&=g}\{-1\}}\alpha(g}\alpha(1-e}\{-0.01\})+g}\alpha(1-e}\{-0.02\}))\textbackslash{}\textbackslash{}
g}\alpha(1-e}\{-0.01\})\&=(-\log \{1-e\textsuperscript{\{-0.01\}\})}2
\textbackslash{} \&=(\log \{1-e\textsuperscript{\{-0.01\}\})}2
\textbackslash{} \&=21.25363\textbackslash{}
g\_\alpha(1-e\textsuperscript{\{-0.02\})\&=(-\log \{1-e}\{-0.02\}\})\^{}2
\textbackslash{} \&=(\log \{1-e\textsuperscript{\{-0.02\}\})}2
\textbackslash{} \&=15.38213\textbackslash{}\textbackslash{}
P(\tau\emph{A\leq 1,\tau\emph{B\leq 1)\&=g\^{}\{-1\}}\alpha(21.25363+15.38213)\textbackslash{}
\&=g\^{}\{-1\}}\alpha(36.63576)\textbackslash{}
\&=\exp(-\sqrt{36.63576})\textbackslash{} \&=0.00235\textbackslash{}
\textbackslash{}end\{split\}

    \hypertarget{ii}{%
\subsection{(ii)}\label{ii}}

\textbackslash{}begin\{split\} P(\tau\_A\leq T \text{ or }
\tau\_B\leq T)\&=P(\tau\_A\leq T)+P(\tau\_B\leq T)-
P(\tau\_A\leq T,\tau\_B\leq T)\textbackslash{}
\&=P(\tau\_A\leq T)+P(\tau\_B\leq T)-C(F\_A(T),F\_B(T))\textbackslash{}
\&=1-e\textsuperscript{\{-0.01\}+1-e}\{-0.02\}-0.00235\textbackslash{}
\&=0.0274\textbackslash{} \textbackslash{}end\{split\}

    \hypertarget{problem-7}{%
\section{Problem 7}\label{problem-7}}

    \textbf{\emph{Continue from the preceding problem. Suppose that a policy
pays 1 million dollars if both A and B default by the end of the first
year. If the interest rate is 0, what would be fair value of this
policy? What if \(\alpha = 1\) instead of \(2\)?}}

    \hypertarget{if-alpha2}{%
\subsubsection{\texorpdfstring{if
\(\alpha=2\)}{if \textbackslash{}alpha=2}}\label{if-alpha2}}

\textbackslash{}begin\{split\} FTD \&=
e\^{}\{-rT\}C(F\_1(T),F\_2(T))\textbackslash{} FTD
\&=1000000\times e\^{}\{-0\times1\}\times0.0023513903\textbackslash{}
\&=2351.3903\textbackslash{} \textbackslash{}end\{split\}

    \hypertarget{if-alpha1}{%
\subsubsection{\texorpdfstring{if
\(\alpha=1\)}{if \textbackslash{}alpha=1}}\label{if-alpha1}}

\[g^{-1}_\alpha(u)=\exp(-u^{1/\alpha})=\exp(-u)\\
\tag{6.1}\] \[
P(\tau_A\leq T,\tau_B\leq T)
=C(F_A(T),F_B(T))\\
\tag{6.2}\] \textbackslash{}begin\{split\} F\_A(T)
\&=1-e\textsuperscript{\{-0.01\}\textbackslash{} F\_B(T)
\&=1-e}\{-0.02\}\textbackslash{}\textbackslash{}
P(\tau\emph{A\leq 1,\tau\emph{B\leq 1)\&=C\_g(F\_A(1),F\_B(1))\textbackslash{}
\&=g\textsuperscript{\{-1\}\emph{\alpha(g}\alpha(F\_A(1))+g\_\alpha(F\_A(1)))\textbackslash{}
\&=g}\{-1\}}\alpha(g}\alpha(1-e\textsuperscript{\{-0.01\})+g\_\alpha(1-e}\{-0.02\}))\textbackslash{}\textbackslash{}
g\_\alpha(1-e\textsuperscript{\{-0.01\})\&=-\log \{1-e}\{-0.01\}\}
\textbackslash{} \&=4.610166\textbackslash{}
g\_\alpha(1-e\textsuperscript{\{-0.02\})\&=-\log \{1-e}\{-0.02\}\}
\textbackslash{} \&=3.922006\textbackslash{}\textbackslash{}
P(\tau\emph{A\leq 1,\tau\emph{B\leq 1)\&=g\^{}\{-1\}}\alpha(4.610166+3.922006)\textbackslash{}
\&=g\^{}\{-1\}}\alpha(8.532172)\textbackslash{}
\&=\exp(-8.532172)\textbackslash{} \&=0.00019702649\textbackslash{}
\textbackslash{}end\{split\}

    Then we can compute the fair value. \textbackslash{}begin\{split\} FTD
\&= e\^{}\{-rT\}C(F\_1(T),F\_2(T))\textbackslash{} FTD
\&=1000000\times e\^{}\{-0\times1\}\times0.00019702649\textbackslash{}
\&=197.0264\textbackslash{} \textbackslash{}end\{split\}


    % Add a bibliography block to the postdoc
    
    
    
    \end{document}
