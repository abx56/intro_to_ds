
% Default to the notebook output style

    


% Inherit from the specified cell style.




    
\documentclass[11pt]{article}

    
    
    \usepackage[T1]{fontenc}
    % Nicer default font (+ math font) than Computer Modern for most use cases
    \usepackage{mathpazo}

    % Basic figure setup, for now with no caption control since it's done
    % automatically by Pandoc (which extracts ![](path) syntax from Markdown).
    \usepackage{graphicx}
    % We will generate all images so they have a width \maxwidth. This means
    % that they will get their normal width if they fit onto the page, but
    % are scaled down if they would overflow the margins.
    \makeatletter
    \def\maxwidth{\ifdim\Gin@nat@width>\linewidth\linewidth
    \else\Gin@nat@width\fi}
    \makeatother
    \let\Oldincludegraphics\includegraphics
    % Set max figure width to be 80% of text width, for now hardcoded.
    \renewcommand{\includegraphics}[1]{\Oldincludegraphics[width=.8\maxwidth]{#1}}
    % Ensure that by default, figures have no caption (until we provide a
    % proper Figure object with a Caption API and a way to capture that
    % in the conversion process - todo).
    \usepackage{caption}
    \DeclareCaptionLabelFormat{nolabel}{}
    \captionsetup{labelformat=nolabel}

    \usepackage{adjustbox} % Used to constrain images to a maximum size 
    \usepackage{xcolor} % Allow colors to be defined
    \usepackage{enumerate} % Needed for markdown enumerations to work
    \usepackage{geometry} % Used to adjust the document margins
    \usepackage{amsmath} % Equations
    \usepackage{amssymb} % Equations
    \usepackage{textcomp} % defines textquotesingle
    % Hack from http://tex.stackexchange.com/a/47451/13684:
    \AtBeginDocument{%
        \def\PYZsq{\textquotesingle}% Upright quotes in Pygmentized code
    }
    \usepackage{upquote} % Upright quotes for verbatim code
    \usepackage{eurosym} % defines \euro
    \usepackage[mathletters]{ucs} % Extended unicode (utf-8) support
    \usepackage[utf8x]{inputenc} % Allow utf-8 characters in the tex document
    \usepackage{fancyvrb} % verbatim replacement that allows latex
    \usepackage{grffile} % extends the file name processing of package graphics 
                         % to support a larger range 
    % The hyperref package gives us a pdf with properly built
    % internal navigation ('pdf bookmarks' for the table of contents,
    % internal cross-reference links, web links for URLs, etc.)
    \usepackage{hyperref}
    \usepackage{longtable} % longtable support required by pandoc >1.10
    \usepackage{booktabs}  % table support for pandoc > 1.12.2
    \usepackage[inline]{enumitem} % IRkernel/repr support (it uses the enumerate* environment)
    \usepackage[normalem]{ulem} % ulem is needed to support strikethroughs (\sout)
                                % normalem makes italics be italics, not underlines
    

    
    
    % Colors for the hyperref package
    \definecolor{urlcolor}{rgb}{0,.145,.698}
    \definecolor{linkcolor}{rgb}{.71,0.21,0.01}
    \definecolor{citecolor}{rgb}{.12,.54,.11}

    % ANSI colors
    \definecolor{ansi-black}{HTML}{3E424D}
    \definecolor{ansi-black-intense}{HTML}{282C36}
    \definecolor{ansi-red}{HTML}{E75C58}
    \definecolor{ansi-red-intense}{HTML}{B22B31}
    \definecolor{ansi-green}{HTML}{00A250}
    \definecolor{ansi-green-intense}{HTML}{007427}
    \definecolor{ansi-yellow}{HTML}{DDB62B}
    \definecolor{ansi-yellow-intense}{HTML}{B27D12}
    \definecolor{ansi-blue}{HTML}{208FFB}
    \definecolor{ansi-blue-intense}{HTML}{0065CA}
    \definecolor{ansi-magenta}{HTML}{D160C4}
    \definecolor{ansi-magenta-intense}{HTML}{A03196}
    \definecolor{ansi-cyan}{HTML}{60C6C8}
    \definecolor{ansi-cyan-intense}{HTML}{258F8F}
    \definecolor{ansi-white}{HTML}{C5C1B4}
    \definecolor{ansi-white-intense}{HTML}{A1A6B2}

    % commands and environments needed by pandoc snippets
    % extracted from the output of `pandoc -s`
    \providecommand{\tightlist}{%
      \setlength{\itemsep}{0pt}\setlength{\parskip}{0pt}}
    \DefineVerbatimEnvironment{Highlighting}{Verbatim}{commandchars=\\\{\}}
    % Add ',fontsize=\small' for more characters per line
    \newenvironment{Shaded}{}{}
    \newcommand{\KeywordTok}[1]{\textcolor[rgb]{0.00,0.44,0.13}{\textbf{{#1}}}}
    \newcommand{\DataTypeTok}[1]{\textcolor[rgb]{0.56,0.13,0.00}{{#1}}}
    \newcommand{\DecValTok}[1]{\textcolor[rgb]{0.25,0.63,0.44}{{#1}}}
    \newcommand{\BaseNTok}[1]{\textcolor[rgb]{0.25,0.63,0.44}{{#1}}}
    \newcommand{\FloatTok}[1]{\textcolor[rgb]{0.25,0.63,0.44}{{#1}}}
    \newcommand{\CharTok}[1]{\textcolor[rgb]{0.25,0.44,0.63}{{#1}}}
    \newcommand{\StringTok}[1]{\textcolor[rgb]{0.25,0.44,0.63}{{#1}}}
    \newcommand{\CommentTok}[1]{\textcolor[rgb]{0.38,0.63,0.69}{\textit{{#1}}}}
    \newcommand{\OtherTok}[1]{\textcolor[rgb]{0.00,0.44,0.13}{{#1}}}
    \newcommand{\AlertTok}[1]{\textcolor[rgb]{1.00,0.00,0.00}{\textbf{{#1}}}}
    \newcommand{\FunctionTok}[1]{\textcolor[rgb]{0.02,0.16,0.49}{{#1}}}
    \newcommand{\RegionMarkerTok}[1]{{#1}}
    \newcommand{\ErrorTok}[1]{\textcolor[rgb]{1.00,0.00,0.00}{\textbf{{#1}}}}
    \newcommand{\NormalTok}[1]{{#1}}
    
    % Additional commands for more recent versions of Pandoc
    \newcommand{\ConstantTok}[1]{\textcolor[rgb]{0.53,0.00,0.00}{{#1}}}
    \newcommand{\SpecialCharTok}[1]{\textcolor[rgb]{0.25,0.44,0.63}{{#1}}}
    \newcommand{\VerbatimStringTok}[1]{\textcolor[rgb]{0.25,0.44,0.63}{{#1}}}
    \newcommand{\SpecialStringTok}[1]{\textcolor[rgb]{0.73,0.40,0.53}{{#1}}}
    \newcommand{\ImportTok}[1]{{#1}}
    \newcommand{\DocumentationTok}[1]{\textcolor[rgb]{0.73,0.13,0.13}{\textit{{#1}}}}
    \newcommand{\AnnotationTok}[1]{\textcolor[rgb]{0.38,0.63,0.69}{\textbf{\textit{{#1}}}}}
    \newcommand{\CommentVarTok}[1]{\textcolor[rgb]{0.38,0.63,0.69}{\textbf{\textit{{#1}}}}}
    \newcommand{\VariableTok}[1]{\textcolor[rgb]{0.10,0.09,0.49}{{#1}}}
    \newcommand{\ControlFlowTok}[1]{\textcolor[rgb]{0.00,0.44,0.13}{\textbf{{#1}}}}
    \newcommand{\OperatorTok}[1]{\textcolor[rgb]{0.40,0.40,0.40}{{#1}}}
    \newcommand{\BuiltInTok}[1]{{#1}}
    \newcommand{\ExtensionTok}[1]{{#1}}
    \newcommand{\PreprocessorTok}[1]{\textcolor[rgb]{0.74,0.48,0.00}{{#1}}}
    \newcommand{\AttributeTok}[1]{\textcolor[rgb]{0.49,0.56,0.16}{{#1}}}
    \newcommand{\InformationTok}[1]{\textcolor[rgb]{0.38,0.63,0.69}{\textbf{\textit{{#1}}}}}
    \newcommand{\WarningTok}[1]{\textcolor[rgb]{0.38,0.63,0.69}{\textbf{\textit{{#1}}}}}
    
    
    % Define a nice break command that doesn't care if a line doesn't already
    % exist.
    \def\br{\hspace*{\fill} \\* }
    % Math Jax compatability definitions
    \def\gt{>}
    \def\lt{<}
    % Document parameters
    \title{Challenge\_Problem}
    
    
    

    % Pygments definitions
    
\makeatletter
\def\PY@reset{\let\PY@it=\relax \let\PY@bf=\relax%
    \let\PY@ul=\relax \let\PY@tc=\relax%
    \let\PY@bc=\relax \let\PY@ff=\relax}
\def\PY@tok#1{\csname PY@tok@#1\endcsname}
\def\PY@toks#1+{\ifx\relax#1\empty\else%
    \PY@tok{#1}\expandafter\PY@toks\fi}
\def\PY@do#1{\PY@bc{\PY@tc{\PY@ul{%
    \PY@it{\PY@bf{\PY@ff{#1}}}}}}}
\def\PY#1#2{\PY@reset\PY@toks#1+\relax+\PY@do{#2}}

\expandafter\def\csname PY@tok@gd\endcsname{\def\PY@tc##1{\textcolor[rgb]{0.63,0.00,0.00}{##1}}}
\expandafter\def\csname PY@tok@gu\endcsname{\let\PY@bf=\textbf\def\PY@tc##1{\textcolor[rgb]{0.50,0.00,0.50}{##1}}}
\expandafter\def\csname PY@tok@gt\endcsname{\def\PY@tc##1{\textcolor[rgb]{0.00,0.27,0.87}{##1}}}
\expandafter\def\csname PY@tok@gs\endcsname{\let\PY@bf=\textbf}
\expandafter\def\csname PY@tok@gr\endcsname{\def\PY@tc##1{\textcolor[rgb]{1.00,0.00,0.00}{##1}}}
\expandafter\def\csname PY@tok@cm\endcsname{\let\PY@it=\textit\def\PY@tc##1{\textcolor[rgb]{0.25,0.50,0.50}{##1}}}
\expandafter\def\csname PY@tok@vg\endcsname{\def\PY@tc##1{\textcolor[rgb]{0.10,0.09,0.49}{##1}}}
\expandafter\def\csname PY@tok@vi\endcsname{\def\PY@tc##1{\textcolor[rgb]{0.10,0.09,0.49}{##1}}}
\expandafter\def\csname PY@tok@vm\endcsname{\def\PY@tc##1{\textcolor[rgb]{0.10,0.09,0.49}{##1}}}
\expandafter\def\csname PY@tok@mh\endcsname{\def\PY@tc##1{\textcolor[rgb]{0.40,0.40,0.40}{##1}}}
\expandafter\def\csname PY@tok@cs\endcsname{\let\PY@it=\textit\def\PY@tc##1{\textcolor[rgb]{0.25,0.50,0.50}{##1}}}
\expandafter\def\csname PY@tok@ge\endcsname{\let\PY@it=\textit}
\expandafter\def\csname PY@tok@vc\endcsname{\def\PY@tc##1{\textcolor[rgb]{0.10,0.09,0.49}{##1}}}
\expandafter\def\csname PY@tok@il\endcsname{\def\PY@tc##1{\textcolor[rgb]{0.40,0.40,0.40}{##1}}}
\expandafter\def\csname PY@tok@go\endcsname{\def\PY@tc##1{\textcolor[rgb]{0.53,0.53,0.53}{##1}}}
\expandafter\def\csname PY@tok@cp\endcsname{\def\PY@tc##1{\textcolor[rgb]{0.74,0.48,0.00}{##1}}}
\expandafter\def\csname PY@tok@gi\endcsname{\def\PY@tc##1{\textcolor[rgb]{0.00,0.63,0.00}{##1}}}
\expandafter\def\csname PY@tok@gh\endcsname{\let\PY@bf=\textbf\def\PY@tc##1{\textcolor[rgb]{0.00,0.00,0.50}{##1}}}
\expandafter\def\csname PY@tok@ni\endcsname{\let\PY@bf=\textbf\def\PY@tc##1{\textcolor[rgb]{0.60,0.60,0.60}{##1}}}
\expandafter\def\csname PY@tok@nl\endcsname{\def\PY@tc##1{\textcolor[rgb]{0.63,0.63,0.00}{##1}}}
\expandafter\def\csname PY@tok@nn\endcsname{\let\PY@bf=\textbf\def\PY@tc##1{\textcolor[rgb]{0.00,0.00,1.00}{##1}}}
\expandafter\def\csname PY@tok@no\endcsname{\def\PY@tc##1{\textcolor[rgb]{0.53,0.00,0.00}{##1}}}
\expandafter\def\csname PY@tok@na\endcsname{\def\PY@tc##1{\textcolor[rgb]{0.49,0.56,0.16}{##1}}}
\expandafter\def\csname PY@tok@nb\endcsname{\def\PY@tc##1{\textcolor[rgb]{0.00,0.50,0.00}{##1}}}
\expandafter\def\csname PY@tok@nc\endcsname{\let\PY@bf=\textbf\def\PY@tc##1{\textcolor[rgb]{0.00,0.00,1.00}{##1}}}
\expandafter\def\csname PY@tok@nd\endcsname{\def\PY@tc##1{\textcolor[rgb]{0.67,0.13,1.00}{##1}}}
\expandafter\def\csname PY@tok@ne\endcsname{\let\PY@bf=\textbf\def\PY@tc##1{\textcolor[rgb]{0.82,0.25,0.23}{##1}}}
\expandafter\def\csname PY@tok@nf\endcsname{\def\PY@tc##1{\textcolor[rgb]{0.00,0.00,1.00}{##1}}}
\expandafter\def\csname PY@tok@si\endcsname{\let\PY@bf=\textbf\def\PY@tc##1{\textcolor[rgb]{0.73,0.40,0.53}{##1}}}
\expandafter\def\csname PY@tok@s2\endcsname{\def\PY@tc##1{\textcolor[rgb]{0.73,0.13,0.13}{##1}}}
\expandafter\def\csname PY@tok@nt\endcsname{\let\PY@bf=\textbf\def\PY@tc##1{\textcolor[rgb]{0.00,0.50,0.00}{##1}}}
\expandafter\def\csname PY@tok@nv\endcsname{\def\PY@tc##1{\textcolor[rgb]{0.10,0.09,0.49}{##1}}}
\expandafter\def\csname PY@tok@s1\endcsname{\def\PY@tc##1{\textcolor[rgb]{0.73,0.13,0.13}{##1}}}
\expandafter\def\csname PY@tok@dl\endcsname{\def\PY@tc##1{\textcolor[rgb]{0.73,0.13,0.13}{##1}}}
\expandafter\def\csname PY@tok@ch\endcsname{\let\PY@it=\textit\def\PY@tc##1{\textcolor[rgb]{0.25,0.50,0.50}{##1}}}
\expandafter\def\csname PY@tok@m\endcsname{\def\PY@tc##1{\textcolor[rgb]{0.40,0.40,0.40}{##1}}}
\expandafter\def\csname PY@tok@gp\endcsname{\let\PY@bf=\textbf\def\PY@tc##1{\textcolor[rgb]{0.00,0.00,0.50}{##1}}}
\expandafter\def\csname PY@tok@sh\endcsname{\def\PY@tc##1{\textcolor[rgb]{0.73,0.13,0.13}{##1}}}
\expandafter\def\csname PY@tok@ow\endcsname{\let\PY@bf=\textbf\def\PY@tc##1{\textcolor[rgb]{0.67,0.13,1.00}{##1}}}
\expandafter\def\csname PY@tok@sx\endcsname{\def\PY@tc##1{\textcolor[rgb]{0.00,0.50,0.00}{##1}}}
\expandafter\def\csname PY@tok@bp\endcsname{\def\PY@tc##1{\textcolor[rgb]{0.00,0.50,0.00}{##1}}}
\expandafter\def\csname PY@tok@c1\endcsname{\let\PY@it=\textit\def\PY@tc##1{\textcolor[rgb]{0.25,0.50,0.50}{##1}}}
\expandafter\def\csname PY@tok@fm\endcsname{\def\PY@tc##1{\textcolor[rgb]{0.00,0.00,1.00}{##1}}}
\expandafter\def\csname PY@tok@o\endcsname{\def\PY@tc##1{\textcolor[rgb]{0.40,0.40,0.40}{##1}}}
\expandafter\def\csname PY@tok@kc\endcsname{\let\PY@bf=\textbf\def\PY@tc##1{\textcolor[rgb]{0.00,0.50,0.00}{##1}}}
\expandafter\def\csname PY@tok@c\endcsname{\let\PY@it=\textit\def\PY@tc##1{\textcolor[rgb]{0.25,0.50,0.50}{##1}}}
\expandafter\def\csname PY@tok@mf\endcsname{\def\PY@tc##1{\textcolor[rgb]{0.40,0.40,0.40}{##1}}}
\expandafter\def\csname PY@tok@err\endcsname{\def\PY@bc##1{\setlength{\fboxsep}{0pt}\fcolorbox[rgb]{1.00,0.00,0.00}{1,1,1}{\strut ##1}}}
\expandafter\def\csname PY@tok@mb\endcsname{\def\PY@tc##1{\textcolor[rgb]{0.40,0.40,0.40}{##1}}}
\expandafter\def\csname PY@tok@ss\endcsname{\def\PY@tc##1{\textcolor[rgb]{0.10,0.09,0.49}{##1}}}
\expandafter\def\csname PY@tok@sr\endcsname{\def\PY@tc##1{\textcolor[rgb]{0.73,0.40,0.53}{##1}}}
\expandafter\def\csname PY@tok@mo\endcsname{\def\PY@tc##1{\textcolor[rgb]{0.40,0.40,0.40}{##1}}}
\expandafter\def\csname PY@tok@kd\endcsname{\let\PY@bf=\textbf\def\PY@tc##1{\textcolor[rgb]{0.00,0.50,0.00}{##1}}}
\expandafter\def\csname PY@tok@mi\endcsname{\def\PY@tc##1{\textcolor[rgb]{0.40,0.40,0.40}{##1}}}
\expandafter\def\csname PY@tok@kn\endcsname{\let\PY@bf=\textbf\def\PY@tc##1{\textcolor[rgb]{0.00,0.50,0.00}{##1}}}
\expandafter\def\csname PY@tok@cpf\endcsname{\let\PY@it=\textit\def\PY@tc##1{\textcolor[rgb]{0.25,0.50,0.50}{##1}}}
\expandafter\def\csname PY@tok@kr\endcsname{\let\PY@bf=\textbf\def\PY@tc##1{\textcolor[rgb]{0.00,0.50,0.00}{##1}}}
\expandafter\def\csname PY@tok@s\endcsname{\def\PY@tc##1{\textcolor[rgb]{0.73,0.13,0.13}{##1}}}
\expandafter\def\csname PY@tok@kp\endcsname{\def\PY@tc##1{\textcolor[rgb]{0.00,0.50,0.00}{##1}}}
\expandafter\def\csname PY@tok@w\endcsname{\def\PY@tc##1{\textcolor[rgb]{0.73,0.73,0.73}{##1}}}
\expandafter\def\csname PY@tok@kt\endcsname{\def\PY@tc##1{\textcolor[rgb]{0.69,0.00,0.25}{##1}}}
\expandafter\def\csname PY@tok@sc\endcsname{\def\PY@tc##1{\textcolor[rgb]{0.73,0.13,0.13}{##1}}}
\expandafter\def\csname PY@tok@sb\endcsname{\def\PY@tc##1{\textcolor[rgb]{0.73,0.13,0.13}{##1}}}
\expandafter\def\csname PY@tok@sa\endcsname{\def\PY@tc##1{\textcolor[rgb]{0.73,0.13,0.13}{##1}}}
\expandafter\def\csname PY@tok@k\endcsname{\let\PY@bf=\textbf\def\PY@tc##1{\textcolor[rgb]{0.00,0.50,0.00}{##1}}}
\expandafter\def\csname PY@tok@se\endcsname{\let\PY@bf=\textbf\def\PY@tc##1{\textcolor[rgb]{0.73,0.40,0.13}{##1}}}
\expandafter\def\csname PY@tok@sd\endcsname{\let\PY@it=\textit\def\PY@tc##1{\textcolor[rgb]{0.73,0.13,0.13}{##1}}}

\def\PYZbs{\char`\\}
\def\PYZus{\char`\_}
\def\PYZob{\char`\{}
\def\PYZcb{\char`\}}
\def\PYZca{\char`\^}
\def\PYZam{\char`\&}
\def\PYZlt{\char`\<}
\def\PYZgt{\char`\>}
\def\PYZsh{\char`\#}
\def\PYZpc{\char`\%}
\def\PYZdl{\char`\$}
\def\PYZhy{\char`\-}
\def\PYZsq{\char`\'}
\def\PYZdq{\char`\"}
\def\PYZti{\char`\~}
% for compatibility with earlier versions
\def\PYZat{@}
\def\PYZlb{[}
\def\PYZrb{]}
\makeatother


    % Exact colors from NB
    \definecolor{incolor}{rgb}{0.0, 0.0, 0.5}
    \definecolor{outcolor}{rgb}{0.545, 0.0, 0.0}



    
    % Prevent overflowing lines due to hard-to-break entities
    \sloppy 
    % Setup hyperref package
    \hypersetup{
      breaklinks=true,  % so long urls are correctly broken across lines
      colorlinks=true,
      urlcolor=urlcolor,
      linkcolor=linkcolor,
      citecolor=citecolor,
      }
    % Slightly bigger margins than the latex defaults
    
    \geometry{verbose,tmargin=1in,bmargin=1in,lmargin=1in,rmargin=1in}
    
    

    \begin{document}
    
    
    \maketitle
    
    

    
    \#

Challenge Problem

Fan Yang

UNI: fy2232

02/28/2018

    \hypertarget{create-data}{%
\subsection{1. Create data}\label{create-data}}

    \begin{Verbatim}[commandchars=\\\{\}]
{\color{incolor}In [{\color{incolor}2}]:} Socio\PYZus{}tab \PY{o}{\PYZlt{}\PYZhy{}} \PY{k+kt}{c}\PY{p}{(}\PY{k+kp}{rep}\PY{p}{(}\PY{l+s}{\PYZdq{}}\PY{l+s}{Low\PYZdq{}}\PY{p}{,}\PY{l+m}{4}\PY{p}{)}\PY{p}{,} \PY{k+kp}{rep}\PY{p}{(}\PY{l+s}{\PYZdq{}}\PY{l+s}{Medium\PYZdq{}}\PY{p}{,}\PY{l+m}{4}\PY{p}{)}\PY{p}{,} \PY{k+kp}{rep}\PY{p}{(}\PY{l+s}{\PYZdq{}}\PY{l+s}{High\PYZdq{}}\PY{p}{,}\PY{l+m}{4}\PY{p}{)}\PY{p}{)}
        Boyscout\PYZus{}tab \PY{o}{\PYZlt{}\PYZhy{}} \PY{k+kp}{rep}\PY{p}{(}\PY{k+kt}{c}\PY{p}{(}\PY{k+kp}{rep}\PY{p}{(}\PY{l+s}{\PYZdq{}}\PY{l+s}{Yes\PYZdq{}}\PY{p}{,}\PY{l+m}{2}\PY{p}{)}\PY{p}{,}\PY{k+kp}{rep}\PY{p}{(}\PY{l+s}{\PYZdq{}}\PY{l+s}{No\PYZdq{}}\PY{p}{,}\PY{l+m}{2}\PY{p}{)}\PY{p}{)}\PY{p}{,}\PY{l+m}{3}\PY{p}{)}
        delinquency\PYZus{}tab \PY{o}{\PYZlt{}\PYZhy{}} \PY{k+kp}{rep}\PY{p}{(}\PY{k+kt}{c}\PY{p}{(}\PY{l+s}{\PYZdq{}}\PY{l+s}{Yes\PYZdq{}}\PY{p}{,}\PY{l+s}{\PYZdq{}}\PY{l+s}{No\PYZdq{}}\PY{p}{)}\PY{p}{,}\PY{l+m}{6}\PY{p}{)}
        frequency \PY{o}{\PYZlt{}\PYZhy{}} \PY{k+kt}{c}\PY{p}{(}\PY{l+m}{10}\PY{p}{,}\PY{l+m}{40}\PY{p}{,}\PY{l+m}{40}\PY{p}{,}\PY{l+m}{160}\PY{p}{,}\PY{l+m}{18}\PY{p}{,}\PY{l+m}{132}\PY{p}{,}\PY{l+m}{18}\PY{p}{,}\PY{l+m}{132}\PY{p}{,}\PY{l+m}{8}\PY{p}{,}\PY{l+m}{192}\PY{p}{,}\PY{l+m}{2}\PY{p}{,}\PY{l+m}{48}\PY{p}{)}
        Socioeconomic \PY{o}{\PYZlt{}\PYZhy{}} \PY{k+kp}{factor}\PY{p}{(}\PY{k+kt}{c}\PY{p}{(}\PY{k+kp}{rep}\PY{p}{(}Socio\PYZus{}tab\PY{p}{[}\PY{l+m}{1}\PY{o}{:}\PY{l+m}{12}\PY{p}{]}\PY{p}{,}frequency\PY{p}{[}\PY{l+m}{1}\PY{o}{:}\PY{l+m}{12}\PY{p}{]}\PY{p}{)}\PY{p}{)}\PY{p}{)}
        Boyscout \PY{o}{\PYZlt{}\PYZhy{}} \PY{k+kp}{factor}\PY{p}{(}\PY{k+kt}{c}\PY{p}{(}\PY{k+kp}{rep}\PY{p}{(}Boyscout\PYZus{}tab\PY{p}{[}\PY{l+m}{1}\PY{o}{:}\PY{l+m}{12}\PY{p}{]}\PY{p}{,}frequency\PY{p}{[}\PY{l+m}{1}\PY{o}{:}\PY{l+m}{12}\PY{p}{]}\PY{p}{)}\PY{p}{)}\PY{p}{)}
        delinquency \PY{o}{\PYZlt{}\PYZhy{}} \PY{k+kp}{factor}\PY{p}{(}\PY{k+kt}{c}\PY{p}{(}\PY{k+kp}{rep}\PY{p}{(}delinquency\PYZus{}tab\PY{p}{[}\PY{l+m}{1}\PY{o}{:}\PY{l+m}{12}\PY{p}{]}\PY{p}{,}frequency\PY{p}{[}\PY{l+m}{1}\PY{o}{:}\PY{l+m}{12}\PY{p}{]}\PY{p}{)}\PY{p}{)}\PY{p}{)}
\end{Verbatim}


    Now we get 3 variables named \textbf{\emph{Socioeconomic}},
\textbf{\emph{Boyscout}} and \textbf{\emph{delinquency}}. They all have
\textbf{800} observations. \emph{Socioeconomic} has 3 levels( ``Low'',
``Medium'' and ``High'') while \emph{Boyscout} and \emph{delinquency}
both have two levels( ``Yes'' and ``No'').

    \hypertarget{exploratory-data-analysis}{%
\subsection{2. exploratory data
analysis}\label{exploratory-data-analysis}}

    \hypertarget{graphical-descriptive-statistics}{%
\subsubsection{Graphical descriptive
statistics}\label{graphical-descriptive-statistics}}

    First let's draw pair-wise comparisons of the three variables.

    The below three plots are pair-wise comparisons of
\emph{``Socioeconomic-Boyscout''}, \emph{``Socioeconomic-delinquency''}
and \emph{``Boyscout-delinquency''}. In each plot, the dark bars
reprensent ``No'' for y-axis while the bright bars represent ``Yes'' for
y-axis. And the width of each bar in each plot stands for the number of
corresponding group. We can easily find that the 3 levels of
Socioeconomic have similar number of observations.

    We can draw conclusion from below plots: (1). For the comparision of
\emph{``Socioeconomic-Boyscout''}, response of Boyscout influenced
significantly by the levels of Socioeconomic. (2). For the other
comparisons, different levels do not show significant differences.

    \begin{Verbatim}[commandchars=\\\{\}]
{\color{incolor}In [{\color{incolor}2}]:} \PY{k+kp}{options}\PY{p}{(}repr.plot.width\PY{o}{=}\PY{l+m}{8}\PY{p}{,} repr.plot.height\PY{o}{=}\PY{l+m}{3}\PY{p}{)}
        par\PY{p}{(}mfrow\PY{o}{=}\PY{k+kt}{c}\PY{p}{(}\PY{l+m}{1}\PY{p}{,}\PY{l+m}{3}\PY{p}{)}\PY{p}{)}
        plot\PY{p}{(}Socioeconomic\PY{p}{,}Boyscout\PY{p}{,}xlab\PY{o}{=}\PY{l+s}{\PYZdq{}}\PY{l+s}{Socioeconomic\PYZdq{}}\PY{p}{,}ylab\PY{o}{=}\PY{l+s}{\PYZdq{}}\PY{l+s}{Boyscout\PYZdq{}}\PY{p}{)}
        plot\PY{p}{(}Socioeconomic\PY{p}{,}delinquency\PY{p}{,}xlab\PY{o}{=}\PY{l+s}{\PYZdq{}}\PY{l+s}{Socioeconomic\PYZdq{}}\PY{p}{,}ylab\PY{o}{=}\PY{l+s}{\PYZdq{}}\PY{l+s}{delinquency\PYZdq{}}\PY{p}{)}
        plot\PY{p}{(}Boyscout\PY{p}{,}delinquency\PY{p}{,}xlab\PY{o}{=}\PY{l+s}{\PYZdq{}}\PY{l+s}{Boyscout\PYZdq{}}\PY{p}{,}ylab\PY{o}{=}\PY{l+s}{\PYZdq{}}\PY{l+s}{delinquency\PYZdq{}}\PY{p}{)}
\end{Verbatim}


    \begin{center}
    \adjustimage{max size={0.9\linewidth}{0.9\paperheight}}{output_9_0.png}
    \end{center}
    { \hspace*{\fill} \\}
    
    \hypertarget{numerical-descriptive-statistics}{%
\subsubsection{Numerical descriptive
statistics}\label{numerical-descriptive-statistics}}

    Denote
\(x_1:\text{Socioeconomic status; } ~x_2:\text{Boy scout; } ~x_3:\text{delinquency.}\)

    Now let's use logistic regression to determine the relationship between
the 3 variables.

    \(\text{logit}(\pi(x_1|x_2,x_3)) = \beta_0 + \beta_1x_2+\beta_2x_3\)

    \textbf{Socioeconomic} vs \textbf{Boyscout} + \textbf{delinquency}

    Since Socioeconomic is a ordinal categorical variable with 3
hierarchies: ``Low'', ``Medium'' and ``High''. We now use
\textbf{multinomial logistic regression model}.

    From the below above, we get the fitted model as: \[
\log\left(\frac{\hat{\pi}_{\text{High}}}{\hat{\pi}_{\text{Low}}}\right)
=-1.1164-0.8038x_2+0.3526x_3\\
\log\left(\frac{\hat{\pi}_{\text{Low}}}{\hat{\pi}_{\text{Medium}}}\right)
=0.2403-0.8038x_2+0.3526x_3
\]

    \begin{Verbatim}[commandchars=\\\{\}]
{\color{incolor}In [{\color{incolor}3}]:} \PY{k+kn}{library}\PY{p}{(}MASS\PY{p}{)}
        polr.cred\PY{o}{\PYZlt{}\PYZhy{}}polr\PY{p}{(}Socioeconomic\PY{o}{\PYZti{}}Boyscout\PY{o}{+}delinquency\PY{p}{)} 
        \PY{k+kp}{summary}\PY{p}{(}polr.cred\PY{p}{)}\PY{o}{\PYZdl{}}coefficients
\end{Verbatim}


    \begin{Verbatim}[commandchars=\\\{\}]

Re-fitting to get Hessian


    \end{Verbatim}

    \begin{tabular}{r|lll}
  & Value & Std. Error & t value\\
\hline
	BoyscoutYes & -0.8037657 & 0.13459568 &  -5.971705\\
	delinquencyYes &  0.3526039 & 0.19354302 &   1.821838\\
	High\textbar{}Low & -1.1163846 & 0.10467675 & -10.665068\\
	Low\textbar{}Medium &  0.2403070 & 0.09648995 &   2.490487\\
\end{tabular}


    
    Use can \textbf{use} \(\mathbf{\chi^2}\) \textbf{test} (
\textbf{Likelihood Ratio Test} ) to check the significance of our model.

    \begin{Verbatim}[commandchars=\\\{\}]
{\color{incolor}In [{\color{incolor}4}]:} drop1\PY{p}{(}polr.cred\PY{p}{,}test\PY{o}{=}\PY{l+s}{\PYZdq{}}\PY{l+s}{Chi\PYZdq{}}\PY{p}{)}
\end{Verbatim}


    \begin{tabular}{r|llll}
  & Df & AIC & LRT & Pr(>Chi)\\
\hline
	<none> & NA           & 1717.873     &        NA    &           NA\\
	Boyscout &  1           & 1752.126     & 36.252338    & 1.733520e-09\\
	delinquency &  1           & 1719.220     &  3.347117    & 6.732282e-02\\
\end{tabular}


    
    Notice that the p-value for coefficients of Boyscout is very small, say
less than 0.01. While p-value for coefficients of delinquency is greater
than 0.05. Therefore under \(\alpha=0.05\) and ``\(H_0\): \(\beta_2=0\)
vs \(H_A\): otherwise'', we fail to reject \(H_0\) and conclude that
\(\beta_2=0\).

    \textbf{use AIC for model selection}

    \begin{Verbatim}[commandchars=\\\{\}]
{\color{incolor}In [{\color{incolor}5}]:} step\PY{p}{(}polr.cred\PY{p}{,}direction \PY{o}{=} \PY{l+s}{\PYZdq{}}\PY{l+s}{backward\PYZdq{}}\PY{p}{)}\PY{o}{\PYZdl{}}anova
\end{Verbatim}


    \begin{Verbatim}[commandchars=\\\{\}]
Start:  AIC=1717.87
Socioeconomic \textasciitilde{} Boyscout + delinquency

              Df    AIC
<none>           1717.9
- delinquency  1 1719.2
- Boyscout     1 1752.1

    \end{Verbatim}

    \begin{tabular}{r|llllll}
 Step & Df & Deviance & Resid. Df & Resid. Dev & AIC\\
\hline
	          & NA       & NA       & 796      & 1709.873 & 1717.873\\
\end{tabular}


    
    In the above output, when we delete delinquency in the model, the AIC is
1719.2 which is very similar to the full model. According to Likelihood
Ratio Test and AIC, we can remove delinquency and refit the model.

    \textbf{refit model}

    \begin{Verbatim}[commandchars=\\\{\}]
{\color{incolor}In [{\color{incolor}6}]:} polr.fit\PY{o}{\PYZlt{}\PYZhy{}}polr\PY{p}{(}Socioeconomic\PY{o}{\PYZti{}}Boyscout\PY{p}{)} 
        \PY{k+kp}{summary}\PY{p}{(}polr.fit\PY{p}{)}\PY{o}{\PYZdl{}}coefficients
\end{Verbatim}


    \begin{Verbatim}[commandchars=\\\{\}]

Re-fitting to get Hessian


    \end{Verbatim}

    \begin{tabular}{r|lll}
  & Value & Std. Error & t value\\
\hline
	BoyscoutYes & -0.8243628 & 0.13416508 &  -6.144392\\
	High\textbar{}Low & -1.1707291 & 0.10052016 & -11.646710\\
	Low\textbar{}Medium &  0.1812920 & 0.09089892 &   1.994435\\
\end{tabular}


    
    From the below above, we get the fitted model as: \[
\log\left(\frac{\hat{\pi}_{\text{High}}}{\hat{\pi}_{\text{Low}}}\right)
=-1.1707291 -0.8243628x_2
\tag{1}\] \[
\log\left(\frac{\hat{\pi}_{\text{Low}}}{\hat{\pi}_{\text{Medium}}}\right)
=0.1812920-0.8243628x_2
\tag{2}\]

    \hypertarget{summary}{%
\subsection{Summary}\label{summary}}

    In this report, we first draw pair-wise comparison of the three
variables. And in view of graph we find that only Boy scout and
Socioeconomic Status have some correlation. The number of levels of
Socioeconomic differ at different levels of Boy scout. Then we conduct
Numerical descriptive statistics. Our main model is \textbf{Multinomial
Logistic Regression Model}. Set response variable as Socioeconomic and
other two as predictor variables. In the first fitted model, we conduct
\(\chi^2\) test to determine

\(H_0\): \(\beta_2=0~\) vs \(~H_A\): otherwise

where \(\beta_2\) is the coefficient of delinquency. Then we get the
p-value is 6.732282e-02 and under \(\alpha=0.05\), we fail to reject
\(H_0\) and conclude that \(\beta_2=0\). Therefore we delete the
variable \textbf{delinquency} and refit the model.

    And get our final model as below:
\[\text{logit}(\pi(x_1|x_2)) = \beta_0 + \beta_1x_2\] where \(\pi()\) is
computed in (1) and (2) and response variable is Socioeconomic and
predictor(\(x_2\)) is Boy scout.


    % Add a bibliography block to the postdoc
    
    
    
    \end{document}
